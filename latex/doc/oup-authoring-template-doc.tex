%%
%% Copyright 2020 OXFORD UNIVERSITY PRESS
%%
%% This file is part of the 'oup-authoring-template Bundle'.
%% ---------------------------------------------
%%
%% It may be distributed under the conditions of the LaTeX Project Public
%% License, either version 1.2 of this license or (at your option) any
%% later version.  The latest version of this license is in
%%    http://www.latex-project.org/lppl.txt
%% and version 1.2 or later is part of all distributions of LaTeX
%% version 1999/12/01 or later.
%%
%% The list of all files belonging to the 'oup-authoring-template Bundle' is
%% given in the file `manifest.txt'.
%%
%% Template article for OXFORD UNIVERSITY PRESS's document class `oup-authoring-template'
%% with bibliographic references
%%
\documentclass{article}

\usepackage{xspace}
\usepackage{fancyvrb}
\usepackage{hyperref}


\begin{document}

\title{\bfseries User Manual\\
OUP Journals---Authoring template}

\author{Copyright Oxford University Press 2020\\
\small Prepared by SPi \TeX\ Support}


\maketitle

\tableofcontents

\section{Introduction }
Oxford University Press has developed this authoring template to
help authors in preparing articles. Authors are encouraged to use
this template to produce \LaTeX\ manuscripts which conform to OUP
Journal styles. This document is a manual for authors to help using
this template during article preparation. It features general
guidelines and contains descriptions regarding various elements that
can be used while preparing manuscripts. Authors are requested to
refer to the file ``readme'' for details of files available for
reference.

Please utilize the OUP authoring template to the maximum, rather
than adding further packages or macros if possible. Please use
semantic mark-up instead of formatting markup where there is a
choice, for instance using \verb+\emph+ rather than \verb+\it+ or
\verb+\textit+ to indicate emphasis.

This documentation is not intended to give an introduction to \LaTeX. For questions concerning \TeX\ systems/installations or the \LaTeX\ mark-up language in general, please visit \url{http://tug.ctan.org/} or any other \TeX\ user group worldwide. The essential reference for \LaTeX\ is Mittelbach F., Goossens M. (2004) \textit{The \LaTeX\ Companion. 2nd edn.}, but there are many other good books about \LaTeX.

\section{Quick reference guide}
\begin{itemize}
\item How to download and install - \url{https://miktex.org/download}
\item Any system requirements - \url{https://miktex.org/kb/prerequisites-2-9}
\item Quick steps to get started - \url{http://users.dickinson.edu/~richesod/latex/latexcheatsheet.pdf}\newline
\url{http://wch.github.io/latexsheet/latexsheet.pdf}\newline
\url{https://www.overleaf.com/learn}
\end{itemize}

\vfill\eject

\section{How to start and prepare your article}
It is assumed that you possess basic knowledge of \LaTeX. Unless
using \LaTeX\ online (for example via Overleaf.com) ensure that you
have \LaTeX2e version installed on your computer. We suggest
employing a recent \TeX installation. You are provided with a class
file ``oup-authoring-template.cls''. This template can be kept with
your manuscript files. Note that the class file depends on the
following packages which are standard and are available along with
\LaTeX installation:

\begin{center}\begin{tabular}{|l|l|l|l|l|}\hline
graphicx & multirow & amsmath & amssymb & amsfonts \\\hline
array & flushend & stfloats & color & xcolor \\\hline
rotating & chngpage & totcount & fix-cm & algorithm \\\hline
algorithmicx & algpseudocode & listings & url & crop \\\hline
\end{tabular}\end{center}

Apart from the above-listed packages, the following additional packages are used in the class file for providing add-on functionalities to the template:

\begin{center}\begin{tabular}{|c|c|c|c|c|}\hline
subfloat & subfig & tikz & hyperref & footnote \\\hline
mathrsfs & natbib & wrapfig & amsthm & apacite \\\hline
\end{tabular}\end{center}

To learn more about the underlying packages, please read the respective documentations (try, e.g., texdoc [package name] at your shell prompt or visit \url{http://tug.ctan.org/}).

\medskip

\noindent You are given with sample tex file: oup-authoring-template.tex

\medskip

Based on your journal style, we would suggest you to use the corresponding sample template file to start your manuscript preparation; other templates available in the package can be removed. Please save a copy of the template file before you start editing the file as per your requirement. These samples contain the lines for calling class files, the preamble area, and the start/end of document where major sample elements required for an article are placed. Comments are included for each element and are self-explanatory. You can add your actual manuscript content in place of these sample elements. The standard structure of each element for an article is explained in detail in the following sections.


To use the OUP authoring template, put all the package files in your
working directory, edit the file ``sample template file'' in your
preferred text editor, and run \LaTeX\ as usual. The resulting
layout is similar but not identical to the layout of the final
article.

Please note that you are not responsible for any final page layout. It is not necessary (and is even sometimes undesirable) to do any fine-tuning with commands like \verb+\break+, \verb+\pagebreak+, \verb+\vspace+, \verb+\clearpage+, etc. Please use semantic mark-up as far as possible and avoid additional formatting commands.

\vfill\eject

\section{Package features and some important settings}

\subsection{Language}
English is the default language used for typesetting rules.

\subsection{Fonts}
Please refrain from using custom fonts.

\medskip

\noindent Text fonts: Unlike the final published version, the authoring template uses non-commercial fonts: Computer Modern. These fonts are free version of the PostScript standard fonts and are supplied as part of all standard \TeX\ distributions.

\medskip

\noindent Math fonts: The standard Computer Modern math fonts are used.

\section{Preamble}
The preamble part comes between the document class line---\newline\verb+\documentclass{...}+---and the beginning of your document---\verb+\begin{document}+. Use this preamble area to include additional packages and customized macros if any.

\section{Details on document class options}

\begin{enumerate}
\item[a.] Options available to select three different types of layout:
\begin{itemize}
\item Modern
\item Traditional
\item Contemporary
\end{itemize}
\item[b.] Options available to select three different paper sizes:
\begin{itemize}
\item Large
\item Medium
\item Small
\end{itemize}
\item[c.] namedate --- for authoryear citation style; default reference citation style is numbered reference style
\item[d.] webpdf --- for cropped paper size in the PDF output
\item[e.] unnumsec --- to get unnumbered section heads
\end{enumerate}

\section{Major structures/elements}

Article contents are divided into three main elements—front matter, main matter, and back matter. The elements preceding the \verb+\maketitle+ tag are considered as front matter elements and the elements placed below the \verb+\maketitle+ tag are main matter elements. Elements found after the section ``Conclusion'' are considered as back matter elements.

\begin{center}\tabcolsep2pt\fontsize{6}{9}\selectfont\begin{tabular}{|l|l|l|}\hline
\textbf{Front matter}&   \textbf{Main matter}&    \textbf{Back matter}\\
\hline
\verb+\title{...}+&     \verb+\section{...}+&   \verb+\begin{appendices}...\end{appendices}+\\\hline
\verb+\author{...}+&    \verb+\subsection{...}+&        \verb+\bibliographystyle{...}+\\\hline
\verb+\authormark{...}+&  \verb+\subsubsection{...}+&     \verb+\bibliography{...}+\\\hline
\verb+\address{...}+&   \verb+\paragraph{...}+& \verb+\begin{biography}{...}{...}\end{biography}+\\\hline
\verb+\corresp[]{...}+&   \verb+\begin{algorithm}...\end{algorithm}+&     \\\hline
\verb+\received{...}{...}{...}+&    \verb+\begin{lstlisting}...\end{lstlisting}+&      \\\hline
\verb+\revised{...}{...}{...}+&     \verb+\begin{table}...\end{table}+&      \\\hline
\verb+\accepted{...}{...}{...}+&    \verb+\begin{figure}...\end{figure}+&    \\\hline
\verb+\abstract{...}+&  \verb+\begin{enumerate}...\end{enumerate}+&      \\\hline
\verb+\keywords{...}+&  \verb+\begin{unlist}...\end{unlist}+&    \\\hline
\verb+\boxedtext{...}+&   \verb+\begin{itemize}...\end{itemize}+&  \\\hline
\verb+\editor{...}+&      \verb+\begin{theorem}...\end{theorem}+&  \\\hline
\verb+ \maketitle+&     \verb+\begin{proposition}...\end{proposition}+&    \\\hline
 &      \verb+\begin{example}...\end{example}+&    \\\hline
 &      \verb+\begin{definition}...\end{definition}+&    \\\hline
 &      \verb+\begin{proof}...\end{proof}+&      \\\hline
&       \verb+\begin{equation}...\end{equation}+&        \\
\hline
\end{tabular}\end{center}


\section{Front matter elements }
Required elements may vary depending on the journal to which you plan to submit. Check the instructions on the journal’s webpage carefully. The tagging details of article opener elements are as follows:
\begin{enumerate}
\item \verb+\title[<short-form-of-article-title>]{<article-title>} +

This tag contains two parameters, first one is optional and the second argument is mandatory. By default, the article title is printed as running heads on both odd/even pages. In case of lengthy article title, provide the short form of article title in the optional argument.

\item \verb+\author[<address-num>]{<author-name>}+---to be used for the authors other than the corresponding author.

\verb+\author[<address-num>,$\ast$]{<author-name>}+---to be used for the corresponding author who is nominated as being responsible for the\break manuscript as it moves through the entire publication process. He is the “time keeper” during each phase of the publication process and the primary contact between the journal and all the other authors of the paper.

\item \verb+\address[<sequence>]{<address-details>}+---affiliation/address details are provided inside this tag. In case of multiple addresses, just provide sequential Arabic numerals in the optional argument of this tag. This numeral is used to denote the affiliation for the respective authors. In case of single author/address, this optional argument can be ignored. For example, refer below:

\verb+\author{...}+

\verb+\address{...}+

\item \verb+\corresp[$\ast$]{...}+— provide corresponding author's email id inside this tag.

\item The other tags listed below are self-explanatory. Note the `received', `revised' and `accepted' dates are placeholders for the final publication: these can be left blank:

{\tabcolsep3pt\fontsize{7}{10}\selectfont\begin{tabular}{|l|l|l|l|l|}
\hline \verb+\orgdiv{. . .}+ & \verb+\orgname{. . .}+ &
\verb+\orgaddress{. . .}+ & \verb+\country{. . .}+ &
\verb+\postcode{. . .}+ \\ \hline \verb+\street{. . .}+ &
\verb+\city{. . .}+ & \verb+\state{. . .}+ & \verb+\abstract{. . .}+
& \verb+\keywords{. . .}+ \\ \hline \verb+\received{. . .}+ &
\verb+\revised{. . .}+ & \verb+\accepted{. . .}+ &  &  \\ \hline
\end{tabular}}

\item \verb+\ORCID{...}+-- provide ORCID link with logo inside this tag.

\item \verb+\maketitle+---this tag is mandatory to print the front matter elements in the output.

\end{enumerate}


\section{Main matter elements }
\subsection{Section headings}
The template allows four levels of headings in different styles:

\medskip

\verb+\section{<First level heading>} +

\verb+\subsection{<Second level heading>} +

\verb+\subsubsection{<Third level heading>}+

\verb+\paragraph{<Fourth level heading>} +

\medskip

\noindent To get unnumbered level heads, provide ``unnumsec'' option to\newline \verb+\documentclass[unnumsec]{oup-authoring-template}+.

\subsection{Mathematical formulae}

As the ``amsmath'' package provides various features for displayed
equations and other mathematical constructs and you are strongly
encouraged to use the mark-ups provided by this package. If
possible, avoid using manual skips to align an equation.

\subsection{Figures and Tables}
The standard interface for graphic inclusion is the \verb+\includegraphics+ command provided by the graphicx package. The ``draft'' option provided by this package locally switches to draft mode, i.e. does not include the graphic, but leaves the correct space, and prints the filename. This option may be used to save the processing time during compilation. Note that the \verb+\graphicspath+ command allows you to declare one or more folders where the graphicx package looks for the image files; hence, it is not necessary to write the path into each \verb+\includegraphics+ command.

The format used for numbered ``figures/tables'' is similar to the basic \LaTeX\ format:

\begin{verbatim}
\begin{figure}[t]
\centering
\includegraphics{<image-file-name>}
\caption{<figure caption text>}\label{...}
\end{figure}
\end{verbatim}

In case of double column layout, the above format puts figure captions/images to a single column width. To get spanned images, we need to provide the environment
\verb+\begin{figure*}...\end{figure*}+.

For the purpose of the sample, we have included the width of images in the optional argument of the \verb+\includegraphics+ tag. Please ignore this.

Images exceeding the text width should be set as rotated images. For this, we need to use \verb+\begin{sidewaysfigure}...\end{sidewaysfigure}+ instead of the \verb+\begin{figure}...\end{figure}+ environment. In case of double column layout, this format puts figure captions/images to single column width. To get spanned rotated images, use\newline \verb+\begin{sidewaysfigure*}...\end{sidewaysfigure*}+.

If the journal requires that you include a graphical abstract, include this as an unnumbered figure directly after the text abstract with no caption, but adding the \verb+\caption+ package to the preamble and including the image as follows (note the file name for the graphical abstract):

\begin{verbatim}
\begin{figure}[t]
\centering
\includegraphics{graphical_abstract.jpg}
\captionsetup{labelformat=empty}
\caption{}
\label{...}
\end{figure}
\end{verbatim}

The format for table is as follows:

\begin{verbatim}
\begin{table}[<float-position>]
\begin{center}
\begin{minipage}{<specify-table-width>}
\caption{...}\label{<table-label>}
\begin{tabular}{<column-alignment-preamble>}
\toprule
...  & ... & ... & ... \\
\midrule
... & ... & ... & ...\footnotemark{1} \\
... & ... & ... & ... \\
... & ... & ... & ...\footnotemark{2} \\
\botrule
\end{tabular}
\footnotetext{...}
\footnotetext[1]{...}
\footnotetext[2]{...}
\end{minipage}
\end{center}
\end{table}
\end{verbatim}

Command to be used for rotated tables:

\begin{verbatim}
\begin{sidewaystable}...\end{sidewaystable}
\end{verbatim}

To span tables across columns in double column layout:

\begin{verbatim}
\begin{table*}...\end{table*}
\end{verbatim}

To span rotated tables across columns in double column layout:

\begin{verbatim}
\begin{sidewaystable*}...\end{sidewaystable*}
\end{verbatim}

\subsection{Lists}
The default list commands available in \LaTeX\ can be used to set different types of lists:

\begin{enumerate}
\item numbered: \verb+\begin{enumerate}...\end{enumerate}+
\item[] first level Arabic numerals;
\item[]second level lowercase alphabet;
\item[]third level lowercase roman numerals;
\item unnumbered: \verb+\begin{unlist}...\end{unlist}+
\item custom list: \verb+\begin{itemize}...\end{itemize}+
\item[]first level bulleted;
\item[]second level bulleted.
\end{enumerate}
Nested lists are allowed for numbered and custom lists.

\subsection{Theorem-like environments}

For theorem-like environments, we require the ``amsthm'' package.
There are three types of predefined theorem styles: thmstyleone,
thmstyletwo, and thmstylethree. The below table shows the output
details for each style:
\begin{center}\begin{tabular}{|l|p{22pc}|}
\hline thmstyleone &Numbered, theorem head in bold font and theorem text in italic style\\
\hline
thmstyletwo & Numbered, theorem head in italic font and theorem text in roman style\\
\hline
thmstylethree & Numbered, theorem head in bold font and theorem text in roman style\\
\hline
\end{tabular}\end{center}
As per the output requirement, the corresponding new theorem style should be defined in the preamble area. For example, if you require a ``Proposition'' environment to be set with ``thmstyletwo,'' then we need to include the below lines in the preamble area:

\begin{verbatim}
\theoremstyle{thmstyletwo}
\newtheorem{proposition}{Proposition}
\end{verbatim}

Refer to the ``amsthm'' package documentation for more details about the additional features available for new theorem styles. As well as the above, a predefined ``proof'' environment is available. This environment prints the ``Proof'' head in italic style and ``body text'' in Roman style with an open square at the end of each proof environment.

\begin{verbatim}
\begin{proof}...\end{proof}
\end{verbatim}

\subsection{Footnotes}

Footnotes are produced with the standard \LaTeX\ command \verb+\footnote{<Some text>}+. This typesets a numerical flag at the location of the footnote command and places the footnote text at the bottom of the page.

\subsection{Algorithms, Program codes, and Listings}

The packages ``algorithm,'' ``algorithmicx,'' and ``algpseudocode'' are used for setting algorithms in \LaTeX. For algorithms, use the below format:

\begin{verbatim}
\begin{algorithm}
\caption{<alg-caption>}\label{<alg-label>}
\begin{algorithmic}[1]
...
\end{algorithmic}
\end{algorithm}
\end{verbatim}
Refer the above-listed package documentation for more details before setting algorithms.

To set program codes, the ``program'' package is required and the command to be used is \verb+\begin{program}...\end{program}+.

The command \verb+\begin{lstlisting}...\end{lstlisting}+ is used to set ``verbatim'' like environments. Refer to the "lstlisting" package documentation for more details.

\subsection{Cross references}

\LaTeX\ is able to automatically insert hypertext links within a document:

\begin{itemize}
\item the \verb+\ref{}+ command adds a clickable link to the referred object;
\item the \verb+\label{}+ command automatically inserts a target;
\end{itemize}

\subsection{Cross citations }
To make a citation in the text, use \verb+\citep{...}+ for a parenthetical citation (Jones et al., 1990), \verb+\citet{...}+ for a textual one, as Jones et al. (1990). Both \verb+\citep+ and \verb+\citet+ are defined by ``natbib'' and are thus not standard. The standard \LaTeX\ command \verb+\cite+ should be avoided, because it behaves like \verb+\citet+ for author–year citations, but like \verb+\citep+ for numerical ones. There also exist the starred versions \verb+\citet*+ and \verb+\citep*+ that print the full author list, and not just the abbreviated one. All of these may take one or two optional arguments to add some text before and after the citation:

\begin{center}\begin{tabular}{|l|l|}\hline
\verb+\citet{jon90}+&   \verb+Jones et al. (1990) +\\\hline
\verb+\citet[chap.~2]{jon90}+&  \verb+Jones et al. (1990, chap. 2) +\\\hline
\verb+\citep{jon90}+&   \verb+(Jones et al., 1990) +\\\hline
\verb+\citep[chap.~2]{jon90}+&  \verb+(Jones et al., 1990, chap. 2) +\\ \hline
\verb+\citep[see][]{jon90}+&    \verb+(see Jones et al., 1990) +\\\hline
\verb+\citep[see][chap.~2]{jon90}+&     \verb+(see Jones et al., 1990, chap. 2) +\\\hline
\verb+\citet*{jon90}+&  \verb+Jones, Baker, and Williams (1990) +\\\hline
\verb+\citep*{jon90}+&  \verb+(Jones, Baker, and Williams, 1990)+\\
\hline
\end{tabular}\end{center}
Please refer to the ``natbib'' package documentation for guidance on other citation commands.

\section{Back matter elements }
\subsection{Appendices}
This section is set with a \verb+\begin{appendices}...\end{appendices}+ environment. All the other commands used to set section heads, tables, and figures inside this section remain the same as the main body text.


\subsection{References}
The basic bibliography environment is accepted for setting a reference section:

\begin{verbatim}
\begin{thebibliography}{9}
\bibitem{bib1} ...
\bibitem{bib2} ...
\end{thebibliography}
\end{verbatim}
However, BiB\TeX\ is the preferred format for references. BiB\TeX\ automates most of the work involved in references for articles. Using BiB\TeX\ options, both citations and references can be automatically updated to the preferred reference style. That is, you need not apply reference style tags for each element manually; it promotes structured writing. Basically, BiB\TeX\ works with two parts of the references: \textit{content} and \textit{style}. The \textit{content} is stored separately in a plain text database file called .bib, in which each entry is structured to distingiush diferent types of entries and fields. The \textit{style} and presentation of the database content are processed with the help of BiB\TeX\ program using a style file called .bst (bibliography style file).

You are requested to use the sample bib file ``reference.bib'' provided as a base for preparing your own .bib file. For the ``author-year'' citation style, use ``abbrvnat'' bst and for ``numbered'' citation style use ``plain'' bst, by adding the following instructions to the preamble.

\begin{enumerate}
\item author-year citation style = \verb+\bibliographystyle{abbrvnat}+
\item numbered citation style = \verb+\bibliographystyle{unsrt}+
\end{enumerate}

Then include your .bib file at the end of your document as shown below:
\begin{verbatim}
\bibliography{<bib-file-without-extension>}
\end{verbatim}
To generate the .bbl file, you should only need to compile \LaTeX/PDF\LaTeX once, then BIB\TeX, then \LaTeX/PDF\LaTeX twice more. The resulting bibliography is ready for typesetting with all formatting tags rendered according to the chosen reference style. For more details, please visit \url{http://www.bibtex.org}.
\section{Author biography}
If this section is required by the journal, please use the below command:
\begin{verbatim}
\begin{biography}
{\includegraphics{<image-file>}}
{\author{<author-Name>} biography-text}
\end{biography}
\end{verbatim}

\section{Author supports}
General support for \LaTeX\ related questions can be obtained from the Internet newsgroup comp.text.tex. Frequently asked questions are available in various web sites dealing with \LaTeX.


\section{Revision History}

\begin{tabular}{|l|l|l|l|}
\hline
Revision State & Revision Date &  Version Number & Revision History\\
\hline
 00 &    16 March, 2020 & Version 1.0   &   \\
 \hline
 01 &    18 March, 2022 & Version 1.1   &   \\
 \hline
\end{tabular}


\end{document}
